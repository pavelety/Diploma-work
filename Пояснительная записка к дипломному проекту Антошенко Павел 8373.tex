\documentclass[12pt,a4paper,onecolumn]{report}
\usepackage[left=2cm,right=1cm,top=2cm,bottom=2cm]{geometry}
\usepackage[utf8x]{inputenc}
\usepackage[english,russian]{babel}
\usepackage{verbatim}
\usepackage{hyperref}
\usepackage{indentfirst}
\usepackage{setspace}
\linespread{1.25}%=1.5 интервал
\usepackage{titlesec}
\titleformat{\chapter}[display]
{\normalfont\huge\bfseries}{\chaptertitlename\ \thechapter}{0pt}{\Huge}
\titlespacing*{\chapter}{0pt}{0pt}{0pt}
\pagestyle{plain}
%\author{Антошенко Павел 8373}
%\title{Пояснительная записка к дипломному проекту}
%\date{г. Санкт-Петербург\\2012 г.}
\begin{document}
\begin{titlepage}
\begin{center}
Министерство образования и науки Российской Федерации\\
Государственное образовательное учреждение\\
Высшего профессионального образования\\
Санкт-Петербургский государственный электротехнический университет «ЛЭТИ» им.~В.И.~Ульянова~(Ленина)\\СПбГЭТУ\\[1cm]

Факультет компьютерных технологий и информатики\\
Кафедра автоматизированных систем обработки информации и управления\\[5cm]
{\large
\textsc{пояснительная записка}\\
к выпускной работе бакалавра на тему:\\
Реализация утилиты для работы со словарем АОТ}
\vfill
г. Санкт-Петербург\\
2012 г.
\end{center}
\end{titlepage}
\tableofcontents
\chapter*{Введение}
\addcontentsline{toc}{chapter}{Введение}
Системы морфологического анализа и синтеза развиваются уже не одно десятилетие, и серьёзная обработка текста уже, пожалуй, немыслима без их помощи. Как в России, так и за рубежом на рынке существуют много коммерческих программ, которые могут успешно справляться с этими задачами, но, к сожалению, они не могут быть использованы для научных экспериментов из-за их крайней высокой цены и отсутствия  исходного кода. С другой стороны, существуют бесплатные модули, которые, впрочем, часто неприемлемы из-за низкой скорости обработки слов и неполноты словарных баз.

Морфологические словари с сайта \href{http://www.aot.ru/}{aot.ru}, которые я использовал, призваны решить указанную выше проблему, обеспечив научные коллективы и вообще любых возможных энтузиастов-экспериментаторов системой морфологического анализа и синтеза.

\chapter{Морфологический словарь morphs.mrd}
\par В данном дипломном проекте используется два морфологических словаря: русский и английский. Русский базируется на грамматическом словаре А.А.~Зализняка 1987 г.\footnote{Основополагающий труд по морфологии, где впервые был предложен системный подход к описанию грамматических парадигм, включающих не только изменение буквенного состава слов, но и ударения.} На данный момент он включает в себя 162519 лемм и 2553 наборов окончаний. Английский словарь создан на основе WordNet\footnote{Cемантическая сеть для английского языка, разработанная в Принстонском университете, и выпущенная вместе с сопутствующим программным обеспечением под некопилефтной свободной лицензией.} и включает в себя 104657 лемм и 442 наборов окончаний. Словари имеют одинаковую структуру~--- они состоят из пяти разделов: наборы окончаний, наборы ударений, история изменений словаря, приставки, леммы. Каждая часть словаря начинается со строки, в которой указано количество строк в данном разделе, что дает возможность итератору вовремя остановиться. Изменяя файл, нужно следить за этими строками.

Первая часть словаря --- это строки с наборами окончаний (флексии) вида: 
\begin{quotation}
<<\%ЫЙ*йа\%ОГО*йб>>,
\end{quotation}

где <<ЫЙ>>, <<ОГО>> --- окончания, а <<йа>>, <<йб>> --- анкоды, обозначающие граммемы леммы (интерпретацию нужно смотреть в rgramtab.tab или egramtab.tab).

Анкодом (<<аношкинским кодом>>) называется уникальный двухбуквенный идентификатор, который соответствует некоторой комбинации значений селективных признаков и граммем. Конечное множество аношкинских кодов исчисляет все встречающиеся в данном языке комбинации морфологических характеристик. Всего в морфологическом анализаторе русского языка системы Диалинг насчитывается 870 таких кодов.

Окончание также может быть и нулевое, например: <<\%*яд>>.

Каждая строка детерминирует отдельную парадигму, поэтому словооснова ссылается на номер строки, соответствующую нужной парадигме.

Следующим разделом идут наборы ударений, которые не используются в моей программе.

Затем идёт блок информации об истории внесения изменений создателями словаря, который также не используется в моей программе.

Далее идут приставки (префиксы), которые подставляются перед словоосновой.

Последним разделом является набор лемм вида:
\begin{quotation}
<<ЯХТСМЕН 51 43 1 Фб ->>,
\end{quotation}

где <<ЯХТСМЕН>> --- словооснова;

<<51>> --- ссылка на набор окончаний (номер строки в разделе окончаний);

<<43>> --- ссылка на набор ударений;

<<1>> --- ссылка на набор приставок;

<<Фб>> --- ссылка на общие граммемы данной леммы (поле Ancode) (может быть <<->>);

<<->> --- не реализовано на данный момент.

Общие граммемы данной леммы, это те граммемы, которые должны быть приписаны всем словоформам данной леммы, например, граммема <<фам>> (фамилия), или граммема «лок» (локативность). Это часто уже семантизированные граммемы.
Набор приставок леммы --- это те приставки, с которыми лемма образует полное слова языка. В набор приставок может входить пустая приставка, что означает, что лемма может быть использована сама по себе (без приставок).

\chapter{Структура файлов словарей rgramtab.tab и egramtab.tab}
Данный словарь служит для определения по анкоду найденного слова части речи, рода, склонения, падежа и т.п.
Состоит он из пустых строк, комментариев (строк, начинающихся с <<//>>) и строк вида: <<аа A С мр,ед,им>>, где\\
<<аа>> --- анкод;\\
<<А>> --- ???;\\
<<С>> --- код части речи (Все обозначения можно узнать в таблице \ref{tabular:partsofspeech});\\
<<мр,ед,им>> --- набор граммем.
\begin{table}[h!]
\caption{Полный перечень русских частей речи.}
\label{tabular:partsofspeech}
\begin{center}
\begin{spacing}{1}
\begin{tabular}{| l | l | l | }
\hline
Часть речи в системе Диалинг & Расшифровка    & Пример\\[2pt] \hline
C             & существительное               & мама\\ \hline
П             & прилагательное                & красный\\ \hline
МС            & местоимение-существительное   & он\\ \hline
Г             & глагол в личной форме         & идет\\ \hline
ПРИЧАСТИЕ     & причастие                     & идущий\\ \hline
ДЕЕПРИЧАСТИЕ  & деепричастие                  & идя\\ \hline
ИНФИНИТИВ     & инфинитив                     & идти\\ \hline
МС-ПРЕДК      & местоимение-предикатив        & нечего\\ \hline
МС-П          & местоименное прилагательное   & всякий\\ \hline
ЧИСЛ          & числительное (количественное) & восемь\\ \hline
ЧИСЛ-П        & порядковое числительное       & восьмой\\ \hline
Н             & наречие                       & круто\\ \hline
ПРЕДК         & предикат                      & интересно\\ \hline
ПРЕДЛ         & предлог                       & под\\ \hline
СОЮЗ          & союз                          & и\\ \hline
МЕЖД          & междометие                    & ой\\ \hline
ЧАСТ          & частица                       & же, бы\\ \hline
ВВОДН         & вводное слово                 & конечно\\ \hline
КР\_ПРИЛ      & краткое прилагательное        & красива\\ \hline
КР\_ПРИЧАСТИЕ & краткое причастие             & построена\\ \hline
\end{tabular}
\end{spacing}
\end{center}
\end{table}
\vskip -1cm
Граммема --- это элементарный морфологический описатель, относящий словоформу к какому-то морфологическому классу, например, словоформе стол с леммой СТОЛ будут приписаны следующие наборы граммем: <<мр, ед, им, но>> и <<мр, ед, вн, но>>. Таким образом, морфологический анализ выдает два варианта анализа  словоформы стол с леммой СТОЛ внутри одной морфологической интерпретации: с винительным <<вн>> и именительным падежами <<им>>.

Ниже перечислены все используемые граммемы:
\begin{description}
\item[мр, жр, ср ---] мужской, женский, средний род;

\item[од, но ---] одушевленность, неодушевленность;

\item[ед, мн ---] единственное, множественное число;

\item[им, рд, дт, вн, тв, пр, зв ---] падежи, соответственно: именительный, родительный, дательный, винительный, творительный, предложный, звательный;

\item[2 ---] обозначает второй родительный или второй предложный падежи;

\item[св, нс ---] совершенный, несовершенный вид;

\item[пе, нп ---] переходный, непереходный глагол;

\item[дст, стр ---] действительный, страдательный залог;

\item[нст, прш, буд ---] настоящее, прошедшее, будущее время;

\item[пвл ---] повелительная форма глагола;

\item[1л, 2л, 3л ---] первое, второе, третье лицо;

\item[0 ---] неизменяемое;

\item[кр ---] краткость (для прилагательных и причастий);

\item[сравн ---] сравнительная форма (для прилагательных);

\item[имя, фам, отч ---] имя, фамилия, отчество;

\item[лок, орг ---] локативность, организация;

\item[кач ---] качественное прилагательное;

\item[вопр,относ ---] вопросительность и относительность (для наречий);

\item[дфст ---] слово обычно не имеет множественного числа;

\item[опч ---] частая опечатка или ошибка;

\item[жарг, арх, проф ---] жаргонизм, архаизм, профессионализм;

\item[аббр ---] аббревиатура;

\item[безл ---] безличный глагол.
\end{description}
Как уже было сказано, одной словоформе может соответствовать много морфологических интерпретаций. Например, у словоформы СТАТЬ две интерпретации:

{СТАТЬ, C, <<но>>, (<<жр,ед,рд>>,<<жр,ед,дт>>, <<жр,мн,им>>, <<жр,мн,вн>>) };

{СТАТЬ, Г, <<нп,св>>,(<<мн,дст,прш>>)}.
\chapter{Работа со словарем}

\chapter{Сравнение производительности}
Работа программы, которая сначала загружает весь словарь в оперативную память и производит считывание и поиск в ней.
Вывод программы, при подаче на вход книги Форда:
Время чтения словаря: 4c 943мс
Общее время: 10c 388мс
Запросов к словарю: 59812. Успешно: 58457
Запросов к словарю в секунду: 10984

Работа программы, которая сначала загружает весь словарь в оперативную память, а затем по ходу выполнения создается кэш-файл также в оперативной памяти, производит считывание и поиск сначала в кэше, если в кэше данной записи еще не существует до ищет и дублирует её из словаря.

 

\chapter{Код}
\verbatiminput{/home/pavlin/Diploma-work/dictionary-reader/src/main/java/com/mycompany/dictionaryreader/analyzer/WordAnalyzer.java}
\end{document}